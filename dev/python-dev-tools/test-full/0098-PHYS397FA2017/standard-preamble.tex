% Beginning code for all standard physics latex documents

%Created on: May 8, 2014    Edited by: Wesley Kyle
%Edited on:	May 12, 2016	Edited by: P. Gimby - cleaned up the code to remove unneeded packages
%Edited on:	May 13, 2016	Edited by: P. Gimby - collected a few more packages used in 325.
%Edited on:	May 16, 2016	Edited by: P. Gimby - fixed page numbering error.
%Edited on: May 20, 2016	Edited by: Alex Shook - Added packages for 497

\documentclass[justified]{tufte-book}
\usepackage{graphicx} % allow embedded images
\setkeys{Gin}{width=\linewidth,totalheight=\textheight,keepaspectratio}
\usepackage{amsmath}  % extended mathematics
\usepackage{bm}  % bold font in math mode
\usepackage{longtable} %lets long tables flow into multiple pages instead of running off the page or having to break tables up manually
\usepackage{booktabs} % book-quality tables
\usepackage{units}    % non-stacked fractions and better unit spacing
\usepackage{multicol} % multiple column layout facilities
\usepackage{tikz} %for drawing nice pictures
\usepackage{indentfirst} % makes first line of each new section be indented
\usepackage{enumitem} % extended options for the enumerate environment
\usepackage{soul} % gives more typestting options like spacing, underline, and strike-through
\usepackage{marvosym} %extra symbols package
\usepackage{multirow} % for special table controls
\usepackage[singlelinecheck=false]{caption} % allow captions w/o figure number
\captionsetup{compatibility=false} % corrects in issue with the caption package
\usepackage{float} % allows for contorl over position of figures and tables
\allowdisplaybreaks % allows equations to span two pages if needed
\usepackage{mathrsfs} % fancy math symbols
\usepackage{multirow} % for special table controls
\usetikzlibrary{arrows,shapes,snakes,calc,patterns,3d} % addon to tikz
\usetikzlibrary{circuits.ee.IEC} % addon to tikz
\usepackage{pgfplots} % package for making plots of functions
\usepackage{gensymb} % symbols i,e. degrees
\usetikzlibrary{decorations.pathmorphing} % to draw the springs
\tikzset{circuit declare symbol = ac source}
\tikzset{set ac source graphic = ac source IEC graphic}
\usepackage{changepage} % allows for full page environment
\usepackage{comment} % allows comment tags for large sections

% define new page style that puts page numbers in the middle
%\begin{comment}
\fancypagestyle{custom}{
\fancyhf{} % clear all header and footer fields
\fancyheadoffset{0pt}
\fancyfootoffset{0pt}
\fancyfoot[C]{\thepage}
\renewcommand{\headrulewidth}{0pt}
\renewcommand{\footrulewidth}{0pt}}
\pagestyle{custom}
%\end{comment}

%below creates a new circuit symbol for AC sources
\tikzset{
         ac source IEC graphic/.style=
          {
           transform shape,
           circuit symbol lines,
           circuit symbol size = width 3 height 3,
           shape=generic circle IEC,
           /pgf/generic circle IEC/before background=
            {
             \pgftransformresetnontranslations
             \pgfpathmoveto{\pgfpoint{-0.8\tikzcircuitssizeunit}{0\tikzcircuitssizeunit}}
             \pgfpathsine{\pgfpoint{0.4\tikzcircuitssizeunit}{0.4\tikzcircuitssizeunit}}
             \pgfpathcosine{\pgfpoint{0.4\tikzcircuitssizeunit}{-0.4\tikzcircuitssizeunit}}
             \pgfpathsine{\pgfpoint{0.4\tikzcircuitssizeunit}{-0.4\tikzcircuitssizeunit}}
             \pgfpathcosine{\pgfpoint{0.4\tikzcircuitssizeunit}{0.4\tikzcircuitssizeunit}}
             \pgfusepathqstroke
            }
          }
        }
% end of circuit symbol
%\begin{document}
%%%end individual beginning code/,$d


%  \begin{titlepage}
%    \vspace*{\fill}
%    \begin{center}
%      \huge{{\bf TITLE1}}\\[0.4cm]
%      \huge{TITLE2}\\[0.4cm]
%      \LARGE{Laboratory Manual}\\[0.4cm]
%      \large{SEASON YEAR}
%    \end{center}
%    \vspace*{\fill}
%  \end{titlepage}
%\maketitle

%\begin{spacing}{0.5}
%\tableofcontents
%\end{spacing}

%NEW PHYS 497 PACKAGES AND COMMANDS

%Subcaption package: Allows subfigures to be placed side by side, and labeled with individual captions (Added June 1, 2016)
\usepackage{subcaption}

%Array package: Allows for addiation specifications in arrays (Added May 6, 2016)
\usepackage{array}

%newcolumntype: Allows one to specify a fixed column width (Added May 6, 2016)
\newcolumntype{L}[1]{>{\raggedright\let\newline\\\arraybackslash\hspace{0pt}}m{#1}}
\newcolumntype{C}[1]{>{\centering\let\newline\\\arraybackslash\hspace{0pt}}m{#1}}
\newcolumntype{R}[1]{>{\raggedleft\let\newline\\\arraybackslash\hspace{0pt}}m{#1}}

%circuits.logic.US, circuits.logic.IEC: For drawing logic gates in Tikz (Added May 6, 2016) 
\usetikzlibrary{circuits.logic.US,circuits.logic.IEC}

\newcommand{\PGT}{ %PGT: positive going transition
\begin{tikzpicture}
\draw[-angle 60] (0,0) -- (0,5pt);
\draw (0,5pt) -- (0,6pt) -- (5pt,6pt);
\draw (-5pt,0) -- (0,0);
\end{tikzpicture}
}





%TEST
\usepackage{geometry}
\pagestyle{fancy}

%\usepackage[caption=false]{subfig}

%\makeatletter
%\renewenvironment{figure}[1][htbp]{%
%  \@tufte@orig@float{figure}[#1]%
%}{%
%  \@tufte@orig@endfloat
%}

%\renewenvironment{table}[1][htbp]{%
%  \@tufte@orig@float{table}[#1]%
%}{%
%  \@tufte@orig@endfloat
%}
%\makeatother

% use instead of subfigure
\makeatletter
\newenvironment{multifigure}[1][htbp]{%
  \@tufte@orig@float{figure}[#1]%
}{%
  \@tufte@orig@endfloat
}
\makeatother

\makeatletter
\newenvironment{mainfigure}[1][htbp]{%
\@tufte@orig@float{figure}[#1]
\begin{adjustwidth}{}{-153pt}}
{\end{adjustwidth}\@tufte@orig@endfloat}%
\makeatother

\makeatletter
\newenvironment{maintable}[1][htbp]{%
\@tufte@orig@float{table}[#1]
\begin{adjustwidth}{}{-153pt}}
{\end{adjustwidth}\@tufte@orig@endfloat}%
\makeatother

%%%% Labatorial Cross-over labs need this code. This should be temporary PG Dec 7, 2016

\newcounter{questioncounter}
\setcounter{questioncounter}{0}
\newcounter{checkpointcounter}
\setcounter{checkpointcounter}{0}
\newcounter{figurecounter}
\setcounter{figurecounter}{0}
%%%%%%%%%%%%%%%%%%%%%%%%%%%%%%%%%%%%%%%%%%%%%%%%%%%%%%%

\newcommand{\checkpoint}{
 \fbox{\begin{minipage}{0.2\textwidth}
 %\includegraphics[width=0.5\textwidth]{stop}
 \end{minipage}
 \begin{minipage}{1.0\textwidth}
 {\bf CHECKPOINT \addtocounter{checkpointcounter}{1} \arabic{checkpointcounter}: Before moving on to the next part, have your TA check the results you obtained so far.}
 \end{minipage}}}

%%% end labatorial cross-over code.

% New environment for placing figure captions under the figure
%\makeatletter
%\newenvironment{mainfigure}{\textwidth}[1][htbp]{%
%\@tufte@orig@float{figure}[#1]%
%}{%
%\@tufte@orig@endfloat
%}
%\makeatother
