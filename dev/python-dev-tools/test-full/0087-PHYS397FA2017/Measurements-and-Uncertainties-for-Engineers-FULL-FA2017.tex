%Edited on: Aug 25, 2016        Edited by: Dr.Kastyak-Ibrahim
%Edited on: 18 July 2017       Edited by: Ania Harlick
%Edited on: Sep 1, 2017         Edited by: Peter Gimby
%Edited on:                     Edited by:
%Edited on:                     Edited by:
%Edited on:                     Edited by:
%Edited on:                     Edited by:
%
%
%

% Beginning code for all standard physics latex documents

%Created on: May 8, 2014    Edited by: Wesley Kyle
%Edited on:	May 12, 2016	Edited by: P. Gimby - cleaned up the code to remove unneeded packages
%Edited on:	May 13, 2016	Edited by: P. Gimby - collected a few more packages used in 325.
%Edited on:	May 16, 2016	Edited by: P. Gimby - fixed page numbering error.
%Edited on: May 20, 2016	Edited by: Alex Shook - Added packages for 497

\documentclass[justified]{tufte-book}
\usepackage{graphicx} % allow embedded images
\setkeys{Gin}{width=\linewidth,totalheight=\textheight,keepaspectratio}
\usepackage{amsmath}  % extended mathematics
\usepackage{bm}  % bold font in math mode
\usepackage{longtable} %lets long tables flow into multiple pages instead of running off the page or having to break tables up manually
\usepackage{booktabs} % book-quality tables
\usepackage{units}    % non-stacked fractions and better unit spacing
\usepackage{multicol} % multiple column layout facilities
\usepackage{tikz} %for drawing nice pictures
\usepackage{indentfirst} % makes first line of each new section be indented
\usepackage{enumitem} % extended options for the enumerate environment
\usepackage{soul} % gives more typestting options like spacing, underline, and strike-through
\usepackage{marvosym} %extra symbols package
\usepackage{multirow} % for special table controls
\usepackage[singlelinecheck=false]{caption} % allow captions w/o figure number
\captionsetup{compatibility=false} % corrects in issue with the caption package
\usepackage{float} % allows for contorl over position of figures and tables
\allowdisplaybreaks % allows equations to span two pages if needed
\usepackage{mathrsfs} % fancy math symbols
\usepackage{multirow} % for special table controls
\usetikzlibrary{arrows,shapes,snakes,calc,patterns,3d} % addon to tikz
\usetikzlibrary{circuits.ee.IEC} % addon to tikz
\usepackage{pgfplots} % package for making plots of functions
\usepackage{gensymb} % symbols i,e. degrees
\usetikzlibrary{decorations.pathmorphing} % to draw the springs
\tikzset{circuit declare symbol = ac source}
\tikzset{set ac source graphic = ac source IEC graphic}
\usepackage{changepage} % allows for full page environment
\usepackage{comment} % allows comment tags for large sections

% define new page style that puts page numbers in the middle
%\begin{comment}
\fancypagestyle{custom}{
\fancyhf{} % clear all header and footer fields
\fancyheadoffset{0pt}
\fancyfootoffset{0pt}
\fancyfoot[C]{\thepage}
\renewcommand{\headrulewidth}{0pt}
\renewcommand{\footrulewidth}{0pt}}
\pagestyle{custom}
%\end{comment}

%below creates a new circuit symbol for AC sources
\tikzset{
         ac source IEC graphic/.style=
          {
           transform shape,
           circuit symbol lines,
           circuit symbol size = width 3 height 3,
           shape=generic circle IEC,
           /pgf/generic circle IEC/before background=
            {
             \pgftransformresetnontranslations
             \pgfpathmoveto{\pgfpoint{-0.8\tikzcircuitssizeunit}{0\tikzcircuitssizeunit}}
             \pgfpathsine{\pgfpoint{0.4\tikzcircuitssizeunit}{0.4\tikzcircuitssizeunit}}
             \pgfpathcosine{\pgfpoint{0.4\tikzcircuitssizeunit}{-0.4\tikzcircuitssizeunit}}
             \pgfpathsine{\pgfpoint{0.4\tikzcircuitssizeunit}{-0.4\tikzcircuitssizeunit}}
             \pgfpathcosine{\pgfpoint{0.4\tikzcircuitssizeunit}{0.4\tikzcircuitssizeunit}}
             \pgfusepathqstroke
            }
          }
        }
% end of circuit symbol
%\begin{document}
%%%end individual beginning code/,$d


%  \begin{titlepage}
%    \vspace*{\fill}
%    \begin{center}
%      \huge{{\bf TITLE1}}\\[0.4cm]
%      \huge{TITLE2}\\[0.4cm]
%      \LARGE{Laboratory Manual}\\[0.4cm]
%      \large{SEASON YEAR}
%    \end{center}
%    \vspace*{\fill}
%  \end{titlepage}
%\maketitle

%\begin{spacing}{0.5}
%\tableofcontents
%\end{spacing}

%NEW PHYS 497 PACKAGES AND COMMANDS

%Subcaption package: Allows subfigures to be placed side by side, and labeled with individual captions (Added June 1, 2016)
\usepackage{subcaption}

%Array package: Allows for addiation specifications in arrays (Added May 6, 2016)
\usepackage{array}

%newcolumntype: Allows one to specify a fixed column width (Added May 6, 2016)
\newcolumntype{L}[1]{>{\raggedright\let\newline\\\arraybackslash\hspace{0pt}}m{#1}}
\newcolumntype{C}[1]{>{\centering\let\newline\\\arraybackslash\hspace{0pt}}m{#1}}
\newcolumntype{R}[1]{>{\raggedleft\let\newline\\\arraybackslash\hspace{0pt}}m{#1}}

%circuits.logic.US, circuits.logic.IEC: For drawing logic gates in Tikz (Added May 6, 2016) 
\usetikzlibrary{circuits.logic.US,circuits.logic.IEC}

\newcommand{\PGT}{ %PGT: positive going transition
\begin{tikzpicture}
\draw[-angle 60] (0,0) -- (0,5pt);
\draw (0,5pt) -- (0,6pt) -- (5pt,6pt);
\draw (-5pt,0) -- (0,0);
\end{tikzpicture}
}





%TEST
\usepackage{geometry}
\pagestyle{fancy}

%\usepackage[caption=false]{subfig}

%\makeatletter
%\renewenvironment{figure}[1][htbp]{%
%  \@tufte@orig@float{figure}[#1]%
%}{%
%  \@tufte@orig@endfloat
%}

%\renewenvironment{table}[1][htbp]{%
%  \@tufte@orig@float{table}[#1]%
%}{%
%  \@tufte@orig@endfloat
%}
%\makeatother

% use instead of subfigure
\makeatletter
\newenvironment{multifigure}[1][htbp]{%
  \@tufte@orig@float{figure}[#1]%
}{%
  \@tufte@orig@endfloat
}
\makeatother

\makeatletter
\newenvironment{mainfigure}[1][htbp]{%
\@tufte@orig@float{figure}[#1]
\begin{adjustwidth}{}{-153pt}}
{\end{adjustwidth}\@tufte@orig@endfloat}%
\makeatother

\makeatletter
\newenvironment{maintable}[1][htbp]{%
\@tufte@orig@float{table}[#1]
\begin{adjustwidth}{}{-153pt}}
{\end{adjustwidth}\@tufte@orig@endfloat}%
\makeatother

%%%% Labatorial Cross-over labs need this code. This should be temporary PG Dec 7, 2016

\newcounter{questioncounter}
\setcounter{questioncounter}{0}
\newcounter{checkpointcounter}
\setcounter{checkpointcounter}{0}
\newcounter{figurecounter}
\setcounter{figurecounter}{0}
%%%%%%%%%%%%%%%%%%%%%%%%%%%%%%%%%%%%%%%%%%%%%%%%%%%%%%%

\newcommand{\checkpoint}{
 \fbox{\begin{minipage}{0.2\textwidth}
 %\includegraphics[width=0.5\textwidth]{stop}
 \end{minipage}
 \begin{minipage}{1.0\textwidth}
 {\bf CHECKPOINT \addtocounter{checkpointcounter}{1} \arabic{checkpointcounter}: Before moving on to the next part, have your TA check the results you obtained so far.}
 \end{minipage}}}

%%% end labatorial cross-over code.

% New environment for placing figure captions under the figure
%\makeatletter
%\newenvironment{mainfigure}{\textwidth}[1][htbp]{%
%\@tufte@orig@float{figure}[#1]%
%}{%
%\@tufte@orig@endfloat
%}
%\makeatother


%\usepackage{graphicx} % allow embedded images
%\setkeys{Gin}{width=\linewidth,totalheight=\textheight,keepaspectratio}
%\usepackage{amsmath}  % extended mathematics
%\usepackage{booktabs} % book-quality tables
%\usepackage{units}    % non-stacked fractions and better unit spacing
%\usepackage{mathrsfs} % fancy math symbols
%\usepackage{booktabs} % book-quality tables
%\usepackage{units}    % non-stacked fractions and better unit spacing
%\usepackage{multicol} % multiple column layout facilities
%\usepackage{fancyvrb} % extended verbatim environments
%\fvset{fontsize=\normalsize}% default font size for fancy-verbatim environments
%\hypersetup{colorlinks} % Comment this line if you don't wish to have colored links
%\usepackage{verbatim} %For creating comment environments
%\usepackage{tikz} %for drawing nice pictures
%\usepackage{bbding} % for the small pencil icon in the record environment below
%\newenvironment{record}{\begin{enumerate}[\NibRight] \slshape}{\upshape \end{enumerate}} 

\begin{document}


%%%start document%%% DO NOT REMOVE THIS LINE
  
\chapter{Determination of the Density of Wood and Steel}
%\author{}
%\date{Fall \the\year}


%\maketitle
%\setcounter{tocdepth}{1}
%\tableofcontents

\section{Equipment}
\begin{minipage}[t]{0.5\textwidth}
\begin{itemize}%[noitemsep]
\item block of wood
\item metal sphere
\end{itemize}
\end{minipage}
%second column
\begin{minipage}[t]{0.5\textwidth}
\begin{itemize}%[noitemsep]
\item calipers
\item meter stick
\end{itemize}
\end{minipage}



\section{Goals of the experiment}
\begin{itemize}
\item To find the best estimate of the density, $\rho$, of a wooden block and a metal sphere at room temperature.
\item To determine the accuracy of the estimation.
\end{itemize}
\section{Background}

A fundamental property characterizing any material is its density. The average density, $\rho$, of a material of mass, M, which occupies the volume, V, is defined as:

\begin{equation}\label{eq:densityqn}
{\rho}=\frac{M}{V}
\end{equation} 
\

The volume of the wooden block and the steel ball will be calculated based on repeated measurements of its dimension(s). Statistical methods will be used to estimate uncertainties of the measurements of each dimension (Type A uncertainty). The uncertainty of the single measurement of the mass of the block is a Type B uncertainty. The uncertainty of the single measurement of the mass of the metal sphere is also a Type B uncertainty. The combined standard uncertainty of the density will be determined using the law of propagation of uncertainty.



\section{Discussion Exercises}
\begin{itemize}
\item
Find in the internet the Guide to the Expression of Uncertainty in Measurement (you can also see the link to it posted on $D2L$ for your course).
\item
Familiarize yourself with the following terms: 
\begin{itemize}
\item uncertainty and error;
\item type A and B of the uncertainty (be aware how type A and B uncertainty is estimated);
\item standard uncertainty and the symbol used for it;
\item combined standard uncertainty;
\end{itemize}
\item In the laboratory report, briefly explain the difference between terms ``precision'' and ``accuracy''.
\end{itemize}


\section{The experiment}
\subsection{Measurement of the volume}
\begin{itemize}
\item
Discuss with your group which method(s) to use to find the volume of the block and the sphere.
\item
Determine the volume of both objects using repeated measurements for the dimension(s).  To incorporate into your measurement the likelihood that the object is not perfect, measure the dimension at different points on the object. \emph{Hint}: In order for the standard deviation to truly represent the uncertainty, the readings must be taken with sufficient accuracy (there will be variations in your readings).
\item
In order to estimate the uncertainty in the dimension(s) use Type A uncertainty 
\item
Calculate the volume and estimate its uncertainty. Explain how you arrived at that estimate.
\end{itemize}


\subsection{Measurement of the mass}
Measure the mass of each object on the balance supplied.  Make the measurement as accurately as possible and estimate the uncertainty in a single measurement - for a digital balance the uncertainty is usually of the order of the last displayed digit (e.g. for a reading of $M$=254.4 g the uncertainty $u(M)$ is 0.1 g). 

\section{Data analysis}
Calculate the density with uncertainty. Comment on any assumptions and estimates you made.

\section{Discussion}
Compare your value of the density of the block with the values of wood density provided in Table \ref{table:wood}. 

Compare the density of the ball with the acceptable value of the density of steel. In your report, reference the source of the acceptable value. 

\begin{table}[ht]
    \centering
    \begin{tabular}{|l|c|}
    \hline
        \textbf{Wood} & $\rho_W$ $(\frac{g}{cm^3})$  \\
        \hline
         Bamboo& 0.3 - 0.4\\
         Oak& 0.6-0.9 \\
         White Pine& 0.35-0.50\\
         \hline
    \end{tabular}
    \caption{Density of selected wood species.}
    \label{table:wood}
\end{table}

\subsection{Comparing values within uncertainties}
For two values, $X_1 \pm u(X_1)$ and $X_2\pm u(X_2)$, to agree within experimental uncertainties, the absolute difference between the values is smaller than the sum of their uncertainties, ie.

\begin{equation}
\label{eq:unc_agreement}
|X_1-X_2|\leq(u(X_1)+u(X_2))
\end{equation}

%If you measured the area of the rectangle, A, to be 1.3 cm$^{2}$ and the standard uncertainty $\delta$A to be 0.6 $cm^{2}$, and Bob obtained A=1.5 cm$^{2}$ and u(A)=0.3 cm$^{2}$, you can say you obtained the same result to within the experimental uncertainties, because the difference between your results, 0.2 cm$^{2}$, is less than the sum of the uncertainties, 0.9 cm$^{2}$.
\subsection{Discussion questions}
\begin{itemize}
\item
Based on the information provided in Table \ref{table:wood} and logical reasoning, determine the material that the block is made of. 
\item
Find examples of publishes values of density of steel and compare them with your results.
\item
What factors could have affected the values measured for the density of wood/steel?

\end{itemize}
\section{The report}
Your report should include the following:
\begin{itemize}
\item
A title; your name, and those of your collaborators;
\item
Objective of the experiment (a summarizing statement about what you set out to observe);
\item Short description of experimental methods;
\item
Data: all the measured values with their uncertainties.
\item
Analysis: Calculation of density of wood and steel and their uncertainties;
\item
Discussion: 
\begin{itemize}
\item a comparison of the calculated values of density with their expected values;
\item comments on possible sources of errors;
\item classification of errors as systematic and random;
\item answers to the three discussion questions.
\end{itemize}
\item
Your conclusion about the block material and whether the values of steel are equal within the experimental uncertainties; appropriate commentary on the agreement/disagreement.
\end{itemize}

\AtEndDocument{\clearpage\ifodd\value{page}\else\null\clearpage\fi} % forces even page count, for double siding

%%%end document%%% DO NOT REMOVE THIS LINE


\end{document}
