%Edited on: Aug 25, 2016        Edited by: Dr.Kastyak-Ibrahim
%Edited on: 18 July 2017       Edited by: Ania Harlick
%Edited on:                     Edited by:
%Edited on:                     Edited by:
%Edited on:                     Edited by:
%Edited on:                     Edited by:
%Edited on:                     Edited by:
%
%
%
%# When the editing the labs, please update this file with the current information
%# regarding the lab. Under "Lab documents" please included the name of the pdf and
%# tex file used to create the lab. Under "Support Files" please include the names
%# of all figure, pictures, or any other file needed to generate the pdf file from the tex file
%# Under "Student Files" please include the names of all file that the student will need to 
%# complete the lab. Such files include the .cmbl file from vernier, movies, physlets,
%# applets, etc. Each lab should be able to run without an internet connection to
%# the outside world. If a file is not listed in this file then it should be deleted from the
%# directory. 
%#
%
% Labatorial Documents
%	Concave-Mirrors-FA2013.tex
%       Concave-Mirrors-FA2013.pdf
%
% Support Files
%	3-7-mhau.jpg
%	mark.png
%	pattern-and-image.png
%	setup.png
%
% Student Files
%	There are no student files
%

\documentclass[justified]{tufte-handout}

\title{Determination of the density of wood and steel}
\author{}
\date{Fall \the\year}

\usepackage{graphicx} % allow embedded images
\setkeys{Gin}{width=\linewidth,totalheight=\textheight,keepaspectratio}
%\graphicspath{{graphics/}} % set of paths to search for images
\usepackage{amsmath}  % extended mathematics
\usepackage{booktabs} % book-quality tables
\usepackage{units}    % non-stacked fractions and better unit spacing
\usepackage{mathrsfs} % fancy math symbols
\usepackage{booktabs} % book-quality tables
\usepackage{units}    % non-stacked fractions and better unit spacing
\usepackage{multicol} % multiple column layout facilities
%\usepackage{lipsum}   % filler text
\usepackage{fancyvrb} % extended verbatim environments
\fvset{fontsize=\normalsize}% default font size for fancy-verbatim environments
\hypersetup{colorlinks} % Comment this line if you don't wish to have colored links
\usepackage{verbatim} %For creating comment environments

\usepackage{tikz} %for drawing nice pictures

\usepackage{bbding} % for the small pencil icon in the record environment below
\newenvironment{record}{\begin{enumerate}[\NibRight] \slshape}{\upshape \end{enumerate}} %new environment to emphasize what needs to be recorded in the lab.
  
\begin{document}

\maketitle
\setcounter{tocdepth}{1}
\tableofcontents

\section{Equipment}
\begin{minipage}[t]{0.5\textwidth}
\begin{itemize}%[noitemsep]
\item[] block of wood
\item[] metal sphere
\end{itemize}
\end{minipage}
%second column
\begin{minipage}[t]{0.5\textwidth}
\begin{itemize}%[noitemsep]
\item[] pair of calipers;
\item[] meter stick.
\end{itemize}
\end{minipage}



\section{Goals of the experiment}
\begin{itemize}
\item To find the best estimate of the density, $\rho$, of a wooden block and a metal sphere at room temperature.
\item To determine the accuracy of the estimation.
\end{itemize}
\section{Background}

A fundamental property characterizing any material is its density. The average density, $\rho$, of a material of mass, M, which occupies the volume, V, is defined as:

\begin{equation}\label{eq:densityqn}
{\rho}=\frac{M}{V}
\end{equation} 
\

The volume of the wooden block and the steel ball will be calculated based on repeated measurements of its dimension(s). Statistical methods will be used to estimate uncertainties of the measurements of each dimension (Type A uncertainty). The uncertainty of the single measurement of the mass of the block is a Type B uncertainty. The uncertainty of the single measurement of the mass of the metal sphere is also a Type B uncertainty. The combined standard uncertainty of the density will be determined using the law of propagation of uncertainty.



\section{Discussion Exercises}
\begin{itemize}
\item
Find in the internet the Guide to the Expression of Uncertainty in Measurement (you can also see the link to it posted on $D2L$ for your course).
\item
Familiarize yourself with the following terms: 
\begin{itemize}
\item uncertainty and error;
\item type A and B of the uncertainty (be aware how type A and B uncertainty is estimated);
\item standard uncertainty and the symbol used for it;
\item combined standard uncertainty;
\end{itemize}
\item In the laboratory report, briefly explain the difference between terms ``precision'' and ``accuracy''.
\end{itemize}


\section{The experiment}
\subsection{Measurement of the volume}
\begin{itemize}
\item
Discuss with your group which method(s) to use to find the volume of the block and the sphere.
\item
Determine the volume of both objects using repeated measurements for the dimension(s).  To incorporate into your measurement the likelihood that the object is not perfect, measure the dimension at different points on the object. \emph{Hint}: In order for the standard deviation to truly represent the uncertainty, the readings must be taken with sufficient accuracy (there will be variations in your readings).
\item
In order to estimate the uncertainty in the dimension(s) use Type A uncertainty 
\item
Calculate the volume and estimate its uncertainty. Explain how you arrived at that estimate.
\end{itemize}


\subsection{Measurement of the mass}
Measure the mass of each object on the balance supplied.  Make the measurement as accurately as possible and estimate the uncertainty in a single measurement - for a digital balance the uncertainty is usually of the order of the last displayed digit (e.g. for a reading of $M$=254.4 g the uncertainty $u(M)$ is 0.1 g). 

\section{Data analysis}
Calculate the density with uncertainty. Comment on any assumptions and estimates you made.

\section{Discussion}
Compare your value of the density of the block with the values of wood density provided in Table \ref{table:wood}. 

Compare the density of the ball with the acceptable value of the density of steel. In your report, reference the source of the acceptable value. 

\begin{table}[ht]
    \centering
    \begin{tabular}{|l|c|}
    \hline
        \textbf{Wood} & $\rho_W$ $(\frac{g}{cm^3})$  \\
        \hline
         Bamboo& 0.3 - 0.4\\
         Oak& 0.6-0.9 \\
         White Pine& 0.35-0.50\\
         \hline
    \end{tabular}
    \caption{Density of selected wood species.}
    \label{table:wood}
\end{table}

\subsection{Comparing values within uncertainties}
For two values, $X_1 \pm u(X_1)$ and $X_2\pm u(X_2)$, to agree within experimental uncertainties, the absolute difference between the values is smaller than the sum of their uncertainties, ie.

\begin{equation}
\label{eq:unc_agreement}
|X_1-X_2|\leq(u(X_1)+u(X_2))
\end{equation}

%If you measured the area of the rectangle, A, to be 1.3 cm$^{2}$ and the standard uncertainty $\delta$A to be 0.6 $cm^{2}$, and Bob obtained A=1.5 cm$^{2}$ and u(A)=0.3 cm$^{2}$, you can say you obtained the same result to within the experimental uncertainties, because the difference between your results, 0.2 cm$^{2}$, is less than the sum of the uncertainties, 0.9 cm$^{2}$.
\subsection{Discussion questions}
\begin{itemize}
\item
Based on the information provided in Table \ref{table:wood} and logical reasoning, determine the material that the block is made of. 
\item
Find examples of publishes values of density of steel and compare them with your results.
\item
What factors could have affected the values measured for the density of wood/steel?

\end{itemize}
\section{The report}
Your report should include the following:
\begin{itemize}
\item
A title; your name, and those of your collaborators;
\item
Objective of the experiment (a summarizing statement about what you set out to observe);
\item Short description of experimental methods;
\item
Data: all the measured values with their uncertainties.
\item
Analysis: Calculation of density of wood and steel and their uncertainties;
\item
Discussion: 
\begin{itemize}
\item a comparison of the calculated values of density with their expected values;
\item comments on possible sources of errors;
\item classification of errors as systematic and random;
\item answers to the three discussion questions.
\end{itemize}
\item
Your conclusion about the block material and whether the values of steel are equal within the experimental uncertainties; appropriate commentary on the agreement/disagreement.
\end{itemize}

\end{document}
